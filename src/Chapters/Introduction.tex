%!TEX root = ../thesis.tex
\chapter{Introduction}

\label{ch:Intro}

This thesis concerns itself with the extension and improvement of \yauhau{}, a IO optimisation compiler plugin for the Ohua framework.
\yauhau{} wasa first introduced in the paper ``Ÿauhau: Concise Code and Efficient I/O Straight from Dataflow.''~\cite{ErtelGoensAdamEtAl2016}.
In this paper we explained how \yauhau{} transforms the dataflow graph of a program to achieve efficient IO while allowing the programmer to write very concise and straight forward code.
We also alluded to some of the issues that arise from control flow when performing the \yauhau{} transformation and we briefly outlined the solutions that we developed.

In this thesis I want to explain in much more detail the concrete ways in which I handle control flow in \yauhau{} and explain implementation considerations and decisions.
Chapter~\ref{ch:smap-transformation} \texttt{smap}, Chapter~\ref{ch:if-transformation} \texttt{if}.
I also want to introduce the reader to a new generalised concept, called context (Chapter~\ref{ch:Context}), which has now been added to Ohua and allows the correct handling of nested control flow structures.
At the beginning however I will introduce the reader to the underlying system Ohua (Chapter~\ref{ch:Ohua}) and \yauhau{}~\ref{ch:Yauhau}, the latter of which includes a description of the transformation as performed by the \yauhau{} plugin.

Contributions of this thesis are:
\begin{itemize}
    \item A more detailed description including implementation considerations and decisions of the \textbf{if and smap transformations}.
    \item The generalisation of non structurally emergent dataflow graph properties into a concept called \textbf{Context}, its detection and use to handle nested control flow in \yauhau{}.
    \item An optional graph transformation in \yauhau{} used to \textbf{preserve write semantics} in programs using the \yauhau{} batching.
    \item New \textbf{experiments} showing the performance of the \yauhau{} plugin in comparison to the existing technologies Haxl and Muse in \textbf{programs with conditionals and mapping}.
    \item Extensions to the random code generator~\cite{Goens-rand-code-graph} used in the \yauhau{} paper to generate test programs.
    These extenstions enable support for mapping operations in the code generator.
\end{itemize}

%
% This thesis intends to improve on the current implementation of this plugin, provide explanation as to why adjustments to the implemetation are necessary in the first place, document adjustments made to the Ohua core which enable the plugin to function and finally expand on the experimental evaluation started in the paper, providing comparative data for performance of our plugin against frameworks with similar functionality.
%
% \section{Fundamental Technologies}
%
% This section shall serve as a reference for technologies which this thesis and its implementation is based upon.
% After this section I presume the reader to be familiar with these technologies and concepts and shall not provide further explanation.
%
% \subsection{Clojure}
%
% Clojure is a functional, dynamically typed language targeting the JVM  which we use to express the higher level algorithms in Ohua.
% Bejond that I use Clojure to implement most of the compiler internal algorithms, including those described in this thesis.
% There will occasionally be code examples to illustrate certain algorithms I am describing, those will usually be written in the Clojure language.
%
% \subsection{Ohua}
%
% Ohua is an automated parallelization framework which combines low level efficiency as provided by the JVM with high level expressions as found in the clojure programming language.
% In Ohua algorithms are implemented in the expressive clojure language and in terms of so called stateful functions, performant and state heavy java code pieces.
% Algorithms implemented in this way will can be automatically disassembled into a data flow graph and parallelized by the Ohua runtime and scheduler.
