% Chapter 1

\chapter{Introduction} % Main chapter title

\label{ChapterIntro} % For referencing the chapter elsewhere, use \ref{Chapter1}

%-------------------------------------------------

\section{Fundamental Technologies}

This section shall serve as a reference for technologies which this thesis and its implementation is based upon.
After this section I presume the reader to be familiar with these technologies and concepts and shall not provide further explanation.

\subsection{Java}

I presume at this point everyone is familiar with Java ...

\subsection{Clojure}

Clojure is a functional, dynamically typed language targeting the JVM  which we use to express the higher level algorithms in Ohua.
Bejond that I use Clojure to implement most of the compiler internal algorithms, including those described in this thesis.

\subsection{Ohua}

Ohua is an automated parallelization framework which combines low level efficiency as provided by the JVM with high level expressions as found in the clojure programming language.
In Ohua algorithms are implemented in the expressive clojure language and in terms of so called stateful functions, performant and state heavy java code pieces.
Algorithms implemented in this way will can be automatically disassembled into a data flow graph and parallelized by the Ohua runtime and scheduler.
