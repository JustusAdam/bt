% Chapter Template

\chapter{Selected data flow fundamentals} % Main chapter title

\label{ChapterDataFlow} % Change X to a consecutive number; for referencing this chapter elsewhere, use \ref{ChapterX}

%----------------------------------------------------------------------------------------
%	SECTION 1
%----------------------------------------------------------------------------------------

\section{Graph representation}

Bindings:

read

write

\subsection{Implementation}

\subsection{Interpretation}

The aforementioned graph implementation can be interpreted as a directed, acyclic graph.
The direction is given by the flow of data, from output to input.
The graphhas to be acyclic.
This is a restriction currently imposed by the underlying Ohua framework but it is also embraced by the algorithms in this thesis because it allows simpler implementations.
Functions are nodes.
I hereby mean a function as a concrete invocation including input and output bindings.
This is in contrast to \textit{function names} which are simply labels to the node describing its functionality.
Bindings are edges.
Each named binding can represent or indicate multiple or no edges.
Bindings are \textit{write once}, hence any binding may only occur once as a return value but may be used an arbitrary number of times as input value.
As a result all edges represented by a particular binding originate from the same node.
Furthermore the graph must be complete, as in any binding read must have previously been written.
For each time a binding is read it represents an edge from its source node to the reading node.
Thus a binding which is never read creates no edges.
