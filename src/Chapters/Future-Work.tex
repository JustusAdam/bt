%!TEX root = ../thesis.tex
\chapter{Conclusion and Future Work}

\label{ch:future-work}

With the introduction of contexts and exploration of their properties I was able to create a generalised and extensible unifying context unwinding algorithm for preparing graphs for the \yauhau{} transformation.
Should a new context structure be added to Ohua the unified unwinding algorithm can be extended by providing a rewrite for only level of the new context.

A new graph rewrite for automatically introducing necessary \texttt{seq} operators into a program with side effects allows \yauhau{} programs with intuitive side effects in the form of writes to a datasource.

Code generator extensions enable new experiments, which show how \yauhau{} delivers comparable, if not better, performance compared to the similar system Haxl while allowing modularity.

An evaluation of the effect of precomputing branches of conditionals lays a foundation for perhaps future optimisations by the compiler.
Precomputing conditional branches can noticeably decrease the number of performed rounds in a program and thus potentially significantly decrease overall execution time.
However an implementation would require a sophisticated selection algorithm for finding sites where this optimisation is beneficial.


\section{Formalised rewrite rules}

Rewrite rules as applied by both the context unrolling and the batching transformation are not formally verified to be correct.
An aim of future work in this direction, not just from \yauhau{} but the Ohua system in general could be the formulation of a core set of program rewrites, primitives, which are then verified to be correct.
Subsequently the transformations as described in Chapters~\ref{ch:if-transformation},~\ref{ch:smap-transformation} and~\ref{ch:Yauhau} could be described in those abstract terms and verified to be correct and semantics preserving.
We have found inspiration for this in a paper ``Generating Performance Portable Code Using Rewrite Rules: From High-level Functional Expressions to High-performance OpenCL Code''\cite{Steuwer:2015:GPP:2858949.2784754}, where formally verified rules are used to decompose and transform high-level functional expressions.

\section{Subgraph batching}

Because the context rewrites introduce so many additional operators for each unwound node it may be beneficial to batch some of those nodes ahead of time to save unwinding nodes.
The principal approach here is to apply the round detection algorithm to the subgraphs of a context and accumulate many independent, parallel, single fetches into a multi-fetch so then we only have to unwind the fewer multi-fetches and not each individual fetch.
This subgraph batching could also be used to help \yauhau{} to deal with cyclic graphs, which as of yet do not exist.

\section{Pattern based code generation}

In the process of generating the code for the experiments, especially \ifop{}, it became apparent that it is very desirable to create random code with specific characteristics and patterns.
In the particular case of the if experiment it would be desirable to create nodes where branches are either precomputed or on the branch.
Currently this needs to be hardcoded in the generator and switched via a command line parameter.
For future work there should be a way to define ad hoc patterns for how a node, or perhaps even whole subgraphs matching certain requirements should be serialised.
Combine this flexibility with a probability for the pattern to be applied and most of what is now a hard coded transformation of graph nodes could be defined with those rules.
