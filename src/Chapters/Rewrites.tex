\chapter{Handling Context in \yauhau}

\label{ChapterRewrites}

\section{Anterior considerations} % TODO Find a better title

Changes made to the program graph in \yauhau{} naturally succeed the context recognition step.
Changes made may include removing nodes from the program graph, or inserting new nodes.
Particularly dangerous is the insertion of new nodes, since at this stage we are not dealing with a simple list of funtions\footnote{As described in the section about the Ohua IR.} but a labeled graph with information about contexts to functions.
Subsequent compiler steps may wish to access the context information also hence we must preserve it, though at present none such compiler steps are implemented or planned.
Furthermore when inserting nodes we must assign correct context information to the newly inserted nodes.

A simple but inefficient way would be to unlabel and relabel the graph every time we make a significant change.
Since the labeling is not particularly cheap this seams inadvisable.
We therefore presume the rewrite implementer to provide appropriate context labels for inserted nodes himself.

\subsection{Providing appropriate context labels}

Assuring labels for nodes inserted after labeling the graph are correct is a hard problem. % TODO how hard?

In the concrete use case of \yauhau{} however we can leverage additional information about the kind of nodes we insert and derive universal properties for these functions which simplify the labeling process.
